
        \documentclass{resume} % Use the custom resume.cls style

        \begin{document}

        \introduction[
            fullname=Jay Gabhawala,
            email=gabhawa@uwindsor.ca,
            phone=4168756340,
            linkedin=https://www.linkedin.com/in/jay-gabhawala-58a339149/,
            github=https://github.com/jaygabha
        ]

        \summary{ Highly energetic, motivated, up and coming Software Engineer looking for an opportunity to expand my skill-set in web development by working with a team of industry experts and contributing to the growth of the organization. }

        \education{
            
            \educationItem[
                university=University of Windsor,
                graduation=2024,
                grade=84.83,
                program=Master of Applied Computing,
                coursework=Studied advanced Computer Science courses like Advanced Software Engineering{,} Advanced Database{,} Advanced System Programming{,} Advanced Computing Concepts along with some Management and Business courses like Managing Employees and Marketing 
            ]
        
            \educationItem[
                university=Nirma University,
                graduation=2021,
                grade=8.3,
                program=Bachelor of Technology in Computer Engineering,
                coursework=Laid groundwork in the principles of Computer Engineering and Science with courses like Theory of Computing{,} Operating Systems{,} Object Oriented Programming. 
            ]
        
        }

        \skills{
            \skillItem[
            category=Languages,
            skills=Java{,} Python{,} C/C++{,} JavaScript{,} HTML/CSS
        ] \\
        \skillItem[
            category=Database,
            skills=SQL{,} MongoDB{,} Firebase
        ] \\
        \skillItem[
            category=Frameworks and Libraries,
            skills= React{,} Backbone.js{,} Bootstrap{,} Webpack{,} Flask{,} Django{,} Jquery{,} Flutter{,} Pandas{,} Numpy{,} Matplotlib{,} Sklearn
        ] \\
        
        }

        \begin{workSection}{Experience}
            
            \experienceItem[
                company=Crest Data Systems LLP,
                position=Software Engineer 2,
                duration=Jan 2021-Aug2022
            ]
            
            \begin{itemize}
                \itemsep -6pt {}\item As a Software Developer Intern, I contributed to developing the Sophos Splunk apps including TA (Technology Add-on) for ingesting data from Sophos Central APIs into Splunk and designed user-friendly Sophos central dashboard app to provide summaries of data from all Sophos apps into Splunk.\item Developed a robust application for the Sophos Kaseya Central Integration deployed as a Kaseya VSA plugin app.\item In the role of Software Engineer 1, collaborated with cross-functional client teams using agile methodology to deliver successful releases.\item Developed and Maintained the various premium apps under the Splunk Umbrella including Splunk Security Essentials which is a  premium security analysis and monitoring app provided by Splunk.\item As a Software Engineer 2, I Spearheaded migration of the Splunk Security Essentials app from Bitbucket-Jenkins to Gitlab and took part in the migration to React Splunk UI from Bootstrap.
            \end{itemize}
        


        \end{workSection}

        \begin{workSection}{Academic projects}
            
                \customItem[
                    title=Crop and Fertilizer Recommendation Engine,
                    organization=University of Windsor,
                    duration=2022
                ]
                
                \begin{itemize}
                    \itemsep -6pt {}\item Researched in a team of 3 by analyzing relevant crop and fertilizer data to build machine learning models that recommend crops and fertilizers to give maximum yield on the basis of soil content and environmental factors.\item Deployed the machine learning models as a Flask API.\item Created a web application using HTML/CSS and JavaScript for front-end, Flask Python for back-end and MongoDB as a database.
                \end{itemize}
            
                \customItem[
                    title=Anomaly detection on Traffic data,
                    organization=Nirma University,
                    duration=2020
                ]
                
                \begin{itemize}
                    \itemsep -6pt {}\item Researched in a team of 2 to create a 2 stage method for time stamp aware anomaly detection on traffic data and Adaptive Gaussian Mixture model for generation of foreground mask and background video. \item Object detection on Background video using YOLO v3. If the object is detected with more than threshold confidence, it is marked as anomalous and time stamp is calculated using frame rate.
                \end{itemize}
             

        \end{workSection}  
        \begin{workSection}  {Certifications}
            
                    \customItem[
                        title=Crest Achievement Certificate,
                        organization=Crest Data Systems,
                        duration=2022
                    ]
                    
                    \begin{itemize}
                        \itemsep -6pt {}\item Awarded the Crest achievement certificate for exceptional performance on Splunk Apps Project to recognize my contribution in Major release of Splunk Security Essential 3.5.0 within deadline
                    \end{itemize}
                
            
        \end{workSection}

        \end{document}  
    